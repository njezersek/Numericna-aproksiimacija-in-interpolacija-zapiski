\documentclass[a3paper,8pt]{extarticle}
% \usepackage[utf8]{inputenc}
\usepackage[mathletters]{ucs}
\usepackage[utf8x]{inputenc}

\usepackage{fancyhdr}

\usepackage[pdftex]{graphicx} % Required for including pictures
\usepackage[pdftex,linkcolor=black,pdfborder={0 0 0}]{hyperref} % Format links for pdf
\usepackage{calc} % To reset the counter in the document after title page
\usepackage{enumitem} % Includes lists

\usepackage{textcomp}
\usepackage{eurosym}

\usepackage{ dsfont } % font za množice
% tabele
\usepackage{array}
\usepackage{wrapfig}

\usepackage{tikz,forest}
\usetikzlibrary{arrows.meta}

\frenchspacing % No double spacing between sentences
\setlength{\parindent}{0pt}
\setlength{\parskip}{0.3em}

\usepackage{mathtools}
\usepackage{blkarray, bigstrut} %

\usepackage{amssymb,amsmath,amsthm,amsfonts}
\usepackage{multicol,multirow}
\usepackage{calc}
\usepackage{ifthen}
\usepackage{tabularx}
\usepackage[landscape]{geometry}
\usepackage{listings}
\usepackage{inconsolata}
%\usepackage[colorlinks=true,citecolor=blue,linkcolor=blue]{hyperref}
%\usepackage{accents}
\usepackage{pdfpages}

\newcommand{\vect}[1]{\accentset{\rightharpoonup}{#1}}

\ifthenelse{\lengthtest { \paperwidth = 11in}}
    { \geometry{top=.5in,left=.5in,right=.5in,bottom=.5in} }
	{\ifthenelse{ \lengthtest{ \paperwidth = 297mm}}
		{\geometry{top=1cm,left=1cm,right=1cm,bottom=1cm} }
		{\geometry{top=1cm,left=1cm,right=1cm,bottom=1cm} }
	}
\pagestyle{empty}
\makeatletter
\renewcommand{\section}{\@startsection{section}{1}{0mm}%
                                {-1ex plus -.5ex minus -.2ex}%
                                {0.5ex plus .2ex}%x
                                {\normalfont\large\bfseries}}
\renewcommand{\subsection}{\@startsection{subsection}{2}{0mm}%
                                {-1explus -.5ex minus -.2ex}%
                                {0.5ex plus .2ex}%
                                {\normalfont\normalsize\bfseries}}
\renewcommand{\subsubsection}{\@startsection{subsubsection}{3}{0mm}%
                                {-1ex plus -.5ex minus -.2ex}%
                                {1ex plus .2ex}%
                                {\normalfont\small\bfseries}}
\makeatother
\setcounter{secnumdepth}{0}
%\setlength{\parindent}{0pt}
%\setlength{\parskip}{0pt plus 0.5ex}

% listings okolje za psevdo kodo
\lstnewenvironment{koda}[1][] %defines the algorithm listing environment
{   
    \lstset{ %this is the stype
        mathescape=true,
        basicstyle=\small, 
		columns=flexible,
        keywordstyle=\bfseries\em,
        keywords={,vhod, izhod, zacetek, konec, koncamo, ponavljaj, dokler, ce, vrni, za, vsak, vse, v, sicer, alg, def} %add the keywords you want, or load a language as Rubens explains in his comment above.
        xleftmargin=.1\textwidth,
		tabsize=4,
		%frame=leftline,xleftmargin=5pt,xrightmargin=5pt,framesep=5pt,
		%inputencoding = utf8,
		extendedchars = true,
		literate={ž}{{\ˇz}}1 {š}{{\ˇs}}1 {č}{{\ˇc}}1 {Ž}{{\ˇZ}}1 {Š}{{\ˇS}}1 {Č}{{\ˇC}}1,
        #1 % this is to add specific settings to an usage of this environment (for instnce, the caption and referable label)
    }
}
{}
% -----------------------------------------------------------------------
\begin{document} 

\begin{multicols}{5}
\setlength{\premulticols}{1pt}
\setlength{\postmulticols}{1pt}
\setlength{\multicolsep}{1pt}
\setlength{\columnsep}{2pt}

\section{Uporabne formule}
\[ (x+y)^n = \sum_{k=0}^{n} \binom{n}{k} x^{n-k}y^{k} \]

\[\binom{n}{k} = \frac{n^{\underline{k}}}{k!} = \frac{n!}{k!(n-k)!} = \binom{n}{n-k}\]

Funkcija $\Gamma$
\begin{itemize}
    \item $ \Gamma(s) = \int_0^{\infty} x^{s-1} e^{-x} dx, \qquad \forall s > 0 $
    \item $\Gamma(1) = 1$
    \item $\Gamma(s+1) = s \Gamma(s)$
    \item $\Gamma(\frac{1}{2}) = \sqrt{\pi}$
    \item $\Gamma(n+1) = n!\qquad n\in \mathbb{N}$
    \item $\Gamma(x)\Gamma(x+1) = \frac{\pi}{\sin(\pi x)}$
\end{itemize}

Funkcija $B$
\begin{itemize}
    \item $B(p,q) = \int_0^1 x^{p-1} (1-x)^{q-1} dx, \qquad \forall p,q > 0$
    \item $\displaystyle B(p,q) = \int_0^{\infty} \frac{u^{p-1}}{(1+u)^{p+q}}du $
    \item $\displaystyle B(p,q) = \frac{\Gamma(p) \Gamma(q)}{\Gamma(p+q)}$
    \item $\frac{1}{2} B(p,q) = \displaystyle \int_0^{\frac{\pi}{2}} (\sin x)^{2p-1} (\cos x)^{2q-1}$
    \item simetričnost: $B(p,q) = B(q,p)$
\end{itemize}

\section*{Aproksimacija}
\subsection{Bernsteinovi polinomi}

\begin{align*}
	B_i^n(x) &= \binom{n}{i} x^i (1-x)^{n-i} \qquad & i = 0, \dots, n \\
	B_i^n(x) &= \sum_{j=i}^n (-1)^{j-i} \binom{n}{j} \binom{j}{i} x^j \qquad & i = 0, \dots, n \\
	B_i^n(x) &= (1-x) B_i^{n-1}(x) + x B_{i-1}^{n-1}(x)  \\ 
\end{align*}

Lastnosti:
\begin{align*}
	B_i^n(x) &\geq 0 \qquad \forall x \in [0, 1] \\
	B_i^n(x) &= B_{n-i}^n(1-x) \\
	\left[B_i^n(x)\right]' &= n \left( B_{i-1}^{n-1}(x) - B_i^{n-1}(x)\right) \\
	\sum_{i=0}^n B_i^n(x) &= 1 \\
	\int B_i^n(x) dx &= \frac{1}{n+1} \sum_{j=i+1}^{n+1} B_j^{n+1}(x) \\
	\int_0^1 B_i^n(x) dx &= \frac{1}{n+1}
\end{align*}

\textbf{Bernsteinov operator} funkciji $f: [0, 1] \to \mathbb{R}$ priredi polinom stopnje $\leq n$:
\[ \big(B_n f\big)(x) = \sum_{i=0}^{n} f\left(\frac{i}{n}\right)  B_i^n(x)\]

Odvod:
\[ \big(B_n f\big)'(x) = n\sum_{i=0}^{n-1} \Delta f\left(\frac{i}{n}\right)  B_i^{n-1}(x) \]
kjer je $ \Delta f\left(\frac{i}{n}\right) = f\left(\frac{i+1}{n}\right) - f\left(\frac{i}{n}\right)$.

Konvergenca:
\[\lim_{n \to \infty} \| f - B_n f \|_{\infty, [0,1]} = 0 \qquad \forall f \in \mathcal{C}([0,1]) \]

\textbf{Kantorovičev operator} integrabilni funkciji $f: [0, 1] \to \mathbb{R}$ priredi polinom stopnje $\leq n$:
\[ \big(K_n f\big)(x) = \sum_{i=0}^{n} f_{n,i} B_i^n(x) \qquad f_{n,i} = (n+1) \int_{\frac{i}{n+1}}^{\frac{i+1}{n+1}} f(t) dt \]

Za $F(x) = \int_0^x f(t) dt$ velja:
\[\big(B_{n+1} F \big) = K_n f \]

Konvergenca:
\[ \lim_{n \to \infty} \| f - K_n f \|_{\infty, [0, 1]} = 0 \qquad \forall f \in \mathcal{C}([0,1])\]

% \subsection{Zvezna odsekoma linearna interpolacija}

\subsection*{Element najboljše aproksimacije}
Naj bo $X$ normiran vektorski prostor z normo $\| \cdot \|$ in $S \subset X$ podprostor.

$s^* \in S$ je \textit{el. najboljše aproksimacije} za $x \in X$, če velja:
\[ \| x - s^* \| = \inf_{s \in S} \| x - s \| = \text{dist}(x, S) \]


Prostor $X$ je \textbf{strogo normiran}, če $\forall x, y \in X$:
\[ \| x + y \| = \| x \| + \| y \| \implies \exists \lambda \in \mathbb{R} : \ x = \lambda y  \]

Če je $X$ strogo normiran in v $S \subset X$ obstaja e. n. a. za $x \in X$, potem je ta enolično določen.


Prostor $X$ je strogo normiran $\iff$ je zaprta enotska krogla strogo konveksna.

\begin{align*}
	M \text{ knv.} &\iff \forall x, y \in M, \lambda \in [0, 1] : \ \lambda x + (1-\lambda) y \in M \\
	M \text{ s.k. } &\iff \forall x \neq y \in M, \lambda \in (0, 1) : \lambda x + (1-\lambda) y \in M^\circ
\end{align*}

Normiran vektorski prostor $X$ je \textbf{enakomerno konveksen}, če $\forall \epsilon > 0 : \ \exists \delta > 0$, da za vsak par elementov $f, g \in X$ z $\| f \| = \| g \| = 1$ velja
\[ \left\| \frac{1}{2} (f+g) \right\| > 1 - \delta \implies \| f-g \| < \epsilon \]

\subsection*{Najboljša polinomska enakomerna aproksimacija}
Iščemo polinom najboljše enakomerne aproksimacije $p^*$ stopnje $\leq n$ za $f \in \mathcal{C}([a, b])$. 

\textbf{1. Remesov postopek} 

\begin{koda}
vhod: $f$, $[a, b]$, $n$.
izberi $E_0 = \left\{ x_i\ |\ a \leq x_0 < \dots < x_n+1 \leq b \right\}$
ponavljaj za $k = 0, 1, \dots$
	poisci $p_{E_k}^*$ p.n.e.a. $\text{za}$ $f$ na mnozici $E_k$ tako, da resis 
		$f(x_i) - p_{E_k}^*(x_i) = (-1)^i m$  $\text{za } i = 0, \dots, n+1$
		(spremenljivke so $m$ in koeficienti polinoma $p_{E_k}^*$)
	poisci ekstrem residuala $r_k = f - p_{E_k}^*$
	doloci $u \in [a, b]$, da je $| r_k(u) | = \| r_k \|_{\infty, [a, b]}$
	ce $| r_k(u) | - |m| < \varepsilon$
		vrni $p_{E_k}^*$ 
	doloci $E_{k+1}$ tako, da zamenjas nek $x_i$ z $u$, tako da 
	ohranis alternacijo residuala 
\end{koda}

\textbf{2. Remesov postopek}
Deluje podobno kot 1. le, da množico $E_{k+1}$ določimo na sledeč način:

\begin{align*}
	z_0 &= a \\
	z_i &\in [x_{i-1}, x_i]  \quad \text{ničle residuala } r(z_i) = 0 \\
	z_{n+1} &= b \\
\end{align*}

\[ u_i = \mathop{\text{argmax}}\limits_{u \in [z_i, z_{i+1}]} r(u)\]

\[ E_{k+1} = \left\{ u_i \ | \ a \leq u_0 < \dots < u_n \leq b \right\} \]

\textit{Ideja: } med $z_i$ in $z_{i+1}$ za $i = 0, \dots, n+1$ poiščemo točko $u_i$, v kateri residual doseže ekstrem hkrati pa pazimo na alternacijo residuala.

\subsection*{Posplošitev najboljše polinomske enak. aproks.}

Posplošeni polinom za sistem funkcij $F = \{f_i | i\}$
\[ p = \sum_{i=0}^n \alpha_i f_i \]

\subsubsection*{Haarov pogoj}
Množica zveznih funkcij $F = \{ f_0, \dots, f_n\}$ na $[a, b]$ zadošča \textit{Haarovemu pogoju}, če je za poljuben izbor $n+1$ paroma različnih točk $(x_i)_i$  posplošena Vandermondova matrika $V_F(x)$ obrnljiva:
\[ V_F(x) = \begin{bmatrix}
	f_j(x_i)
\end{bmatrix}_{i,j=0}^n \]

$F$ zadošča Haarovemu pogoju $\implies$ $F$ je linearno neodvisna.

$F$ ki zadošča Haarovemu pogoju, je \textbf{Čebišev sistem funkcij}, če pa zahtevamo Harrov pogoj le za $x_i \in [a, b]$, pa pravimo, da je $F$ \textbf{šibki Čebišev sistem funkcij}. 

Če $F$ zadošča Haarovemu pogoju, lahko za iskanje e. n. a. iz prostora $\text{Lin}(F)$ uporabimo \textbf{Remesov postopek}.

\subsubsection*{Polinomi Čebiševa}
\[ T_n(x) = \cos(n \arccos x) \qquad x \in [-1, 1] \]
Rekurzivna zveza:
\[ T_{n}(x) = 2x T_{n-1}(x) - T_{n-2}(x) \qquad n \geq 2 \]
\[ T_0(x) = 1 \qquad T_1(x) = x \]

$T_n$ ima ničle v $\cos \frac{(2k-1)\pi}{2n}$ za $k = 1, \dots, n$ ekstreme pa doseže v $\cos(\frac{k\pi}{n})$ za $k = 0, \dots, n$.

$p \in \mathbb{P}_{n-1}$, ki na $[-1, 1]$ najbolje aproksimira $\sum_{i=0}^{n} a_i T_i(x)$ je kar $\sum_{i=0}^{n-1} a_i T_i(x)$.

Polinomi Čebiševa tvorijo Čebišev sistem funkcij.

\subsection*{Aproksimacija po metodi najmanjših kvadratov}
Naj bo $X$ vektorski prostor nad $\mathbb{R}$ s skalarnim produktom $\langle \cdot, \cdot \rangle$ in normo $\| \cdot \|_2 = \sqrt{\langle \cdot, \cdot \rangle}$ 
\[ S = \text{Lin}\{l_1, l_2, \dots, l_n\} \subseteq X \]

Iščemo \textbf{element najboljše aproksimacije po MNK} $f^* \in S$, da $\| f - f^* \| = \min_{s \in S} \| f - s \| $

\textit{Izrek:} $f^*$ je el. najboljše aproksimacije po MNK $\iff$ $f-f^* \perp S$ $\iff$
$f-f^* \perp l_i \quad \forall i = 1,\dots n$

\[ f^* = \alpha_1 l_1 + \dots + \alpha_n l_n\]

Iz zgornjega izreka sledi:
\begin{align*}
\langle f - f^*, l_i \rangle &= 0 \quad \forall i\\
\langle f - \sum_{j=1}^n \alpha_j l_j, l_i \rangle &= 0 \quad \forall i\\
\langle f, l_i\rangle - \sum_{j=1}^n \alpha_j \langle  l_j, l_i \rangle &= 0 \quad \forall i\\
\end{align*}
V matrični obliki:
\[
    \underbrace{\begin{bmatrix}
        \langle l_1, l_1 \rangle & \dots & \langle l_n, l_1 \rangle \\
        \vdots & \ddots & \vdots \\
        \langle l_1, l_n \rangle & \dots & \langle l_n, l_n \rangle
    \end{bmatrix}}_{\text{Grammova matrika $G$}}
    \begin{bmatrix}
        \alpha_1 \\
        \vdots \\
        \alpha_n
    \end{bmatrix}
    =
    \begin{bmatrix}
        \langle f, l_1 \rangle \\
        \vdots \\
        \langle f, l_n \rangle
    \end{bmatrix}
\]

$G$ je simetrična pozitivno definitna matrika. Numerično tak sistem rešimo z razcepom Choleskega.


Reševanje sistema linearnih enačb se izognemo tako, da bazo za $S$ ortonormiramo. Tedaj je $G = I$ in
\[ f^* = \sum_{i=1}^n \langle f, l_i \rangle l_i \]

\subsubsection*{Gram-Schmidtova ortogonalizacija}
Definirajmo projekcijo vektorja $v$ na $u$
\[\textmd{proj}_u(v) = \frac{\langle v,u \rangle}{\langle u,u \rangle}u\]
Če želimo \emph{orotogonalizirati} $k$ linearno neodvisnih vektorjev $v_1, ..., v_k$, uporabimo postopek:
\begin{equation*}
    \begin{aligned}
    u_1 &= v_1 \\
    u_2 &= v_2 - \textmd{proj}_{u_1}(v_2) \\
    u_3 &= v_3 - \textmd{proj}_{u_1}(v_3) - \textmd{proj}_{u_2}(v_3) \\
    &\; \ \vdots \\
    u_k &= v_k - \sum_{j=1}^{k-1} \textmd{proj}_{u_j}(v_k)
    \end{aligned}
\end{equation*}

\subsubsection*{Legendrovi polinomi}
\[ P_n(x) = \frac{1}{2^n n!} \frac{d^n}{dx^n} \left[ (x^2 - 1)^n \right] \]

tvorijo ortogonalno bazo za $\mathcal{P}_n$ s glede na skalarni produkt:
\[ \langle g, h  \rangle = \int_{-1}^{1} g(x) h(x) dx \]

\subsubsection{Tričlenska rekurzivna formula}
\begin{gather*}
	\widetilde{Q}_n = (x - \alpha_n) Q_{n-1} - \beta_n Q_{n-2} \qquad Q_n = \frac{\widetilde{Q}_n}{\beta_{n+1}} \\	
	\alpha_n = \langle xQ_{n-1}, Q_{n-1} \rangle \qquad \beta_n = \langle \widehat{Q}_{n-1}, \widehat{Q}_{n-1} \rangle ^{\frac{1}{2}} \\
\end{gather*}

tvori ortonormirane (glede na isti skalarni produkt kot ga uporabljamo pri Legendrovih polinomih) polinome $Q_n$ stopnje $n$.

\section*{Interpolacija}
Naj bo $X$ normiran vektorski prostor z normo $\| \cdot \|$ in $S \subset X$ podprostor tako, da
\[ S = \text{Lin} \{ \varphi_1, \dots , \varphi_n \}\]

Za dane linearne funkcionale $(\lambda_i : X \to \mathbb{R})_{i=1}^n$ in vrednosti $r = (r_i)_{i=1}^n$ iščemo koeficiente $(\alpha_i)_{i=1}^n$, da bo za \textit{interpolant} $p = \sum_{i=1}^n \alpha_i \varphi_i $ veljalo
\[ \lambda_i p = r_i \qquad i, j = 1, \dots, n \]

pri interpolaciji izbranega elementa $f \in X$ določimo
\[ r_i = \lambda_i f \]

Interpolacijski problem je \textbf{korekten}, kadar interpolant obstaja in je enolično določen.

\subsection*{Polinomska interpolacija}
$f \in \mathcal{C}([a,b])$, $a \leq x_0 < \dots < x_n \leq b$. Iščemo polinom
\[ p(x) = a_0 + a_1 x + a_2 x^2 + \dots + a_n x^n \]
ki zadošča pogoju $f(x_i) = p(x_i)$ za $i = 0, 1, \dots, n$. 
\[
\underbrace{\begin{bmatrix}
    1 & x_0 & x_0^2 & \dots & x_0^n \\
    1 & x_1 & x_1^2 & \dots & x_1^n \\
    \vdots & &        &       & \vdots \\
    1 & x_n & x_n^2 & \dots & x_n^n \\
\end{bmatrix}}_{\text{Vandermondova matrika $V$}}
\begin{bmatrix}
    a_0 \\
    a_1 \\
    \vdots \\
    a_n
\end{bmatrix}
=
\begin{bmatrix}
    f(x_0) \\
    f(x_1) \\
    \vdots \\
    f(x_n) \\
\end{bmatrix}
\]
\[ det V = \prod_{0 \leq j < i \leq n} (x_i - x_j) \neq 0\]
Vidimo, da je polinom $p$ enolično določen.

\subsubsection*{Lagrangeova oblika interpolacijskega polinoma}
\[ l_{i, n}(x) = \prod_{\substack{j=0 \\ j\neq i}}^n \frac{x-x_j}{x_i - x_j} \qquad i = 0, \dots, n \]

Velja:
\[ 
l_{i,n}(x_j) = 
\begin{cases}
    1 & i = j \\
    0 & \text{sicer}
\end{cases}
\]
\textit{Lema:} $l_{i, n}$ so baza za $\mathbb{P}_n$.


Interpolacijski polinom v lagrangeovi obliki:
\[ \big( I_n f \big)(x) = p(x) = \sum_{i=0}^n f(x_i) l_{i,n}(x)\]

$I_n$ je linearen operator. Je tudi \textit{projektor} na $\mathbb{P}_n$ ($\forall q \in \mathbb{P}_n : I_n q = q$).

\[ f(x) - p(x) = \frac{f^{(n+1)}(\xi)}{(n+1)!} \omega(x) \]

\[ \omega(x) = \prod_{i=0}^n (x-x_i)\]

Nevilleova shema za izračun $p = p_{0,n}$:
\begin{align*}
	p_{i, 0}(x) &= f(x_i) \\
	p_{i, k}(x) &= p_{i, k-1}(x) \frac{x-x_{i+k}}{x_i + x_{i+k}} + p_{i+1, k-1}(x) \frac{x-x_i}{x_{i+k} - x_i} \\
\end{align*}

\subsubsection*{Newtonova oblika zapisa interpolacijskega polinoma}

Za bazo izberemo prestavljene potence: $1$, $(x-x_0)$, $(x-x_0)(x-x_1)$, \dots  


\textbf{Deljena diferenca} $[x_0, x_1, \dots, x_k] f$ je vodilni koeficient interpolacijskega polinoma
stopnje $k$, ki se s funkcijo $f$ ujema v točkah $x_0, x_1, \dots, x_k$. Sledi:
\[ p_k(x) = p_{k-1}(x) + [x_0, x_1, \dots, x_k] f (x-x_0)\dots (x-x_{k-1})\]
Rekurzivna zveza:
\begin{multline*}
    [x_i, x_{i+1}, \dots, x_{i+k}]f = \\
    = \begin{cases}
        \frac{f^{(k)}(x_i)}{k!} & {\scriptstyle x_i = x_{i+1} = \dots = x_{i+k}} \\
        \frac{[x_{i+1}, \dots, x_{i+k}]f- [x_i, \dots, x_{i+k-1}]f}{x_{i+k}-x_i} & {\scriptstyle x_i \neq x_{i+k} }
    \end{cases} 
\end{multline*} 

Newtonova oblika zapisa je torej:
\[ p(x) = \sum_{i=0}^n [x_0, \dots, x_i]f \cdot (x-x_0)\dots (x-x_{i-1}) \]

\textit{Če želimo, da se polinom n neki toči ujema tudi v $k$-tem odvodu, to točko $k$-krat ponovimo.}

\[ f(x) = p(x) + \omega(x) \underbrace{[x_0, \dots, x_n, x] f}_{\frac{f^{(n+1)}(\xi)}{(n+1)!}} \quad \omega(x) = \prod_{i=0}^n (x-x_i)\]

\textbf{Napaka} interpolacije
\[f(x) - p(x) = [x_0, \dots, x_n, x]f \omega(x) = \omega(x) \frac{f^{(n+1)}(\xi)}{(n+1)!}\]

za nek $\xi \in [a, b]$.

\subsubsection*{Računanje vrednosti interpolacijskega polinoma v Newtonovi obliki}
\[ p(x) = d_0 1 + d_1(x-x_0) + d_2(x-x_0)(x-x_1) + \dots \]

Posplošen Hornerjev algoritem:
\begin{koda}
vhodni podatki: $x_0, \dots, x_n, \ d_0, \dots, d_n, \ x$
$v_n \leftarrow d_n$
za $i = n-1, \dots, 0$:
    $v_i \leftarrow d_i + (x-x_i) v_{i+1}$

vrni $v_0$
\end{koda}



\end{multicols}


\end{document}